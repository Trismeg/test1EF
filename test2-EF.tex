\documentclass{article}

\usepackage{pgfplots}
\usepackage[margin=0.75in, paperwidth=8.5in, paperheight=11in]{geometry}
\usepackage{setspace}
\usepackage{booktabs}
\usepackage[]{units}


% For nicely typeset tabular material
\usepackage{booktabs}

%%
% For graphics / images
%\usepackage{graphicx}
%\usepackage[dvipsnames]{xcolor}
\usepackage[T1]{fontenc} % Use 8-bit encoding that has 256 glyphs
\usepackage{fourier} % Use the Adobe Utopia font for the document - comment this line to return to the LaTeX default
\usepackage[english]{babel} % English language/hyphenation
\usepackage{amsmath,amsfonts,amsthm} % Math packages
\usepackage{mathtools}% http://ctan.org/pkg/mathtools
\usepackage{etoolbox}% http://ctan.org/pkg/etoolbox


\usepackage{lipsum} % Used for inserting dummy 'Lorem ipsum' text into the template


\usepackage{units}

\usepackage{cancel}
\usepackage{physymb}




\usepackage{sectsty} % Allows customizing section commands
%\usepackage[dvipsnames]{xcolor}
\usepackage{pgf,tikz}
\usetikzlibrary{decorations.markings}
\usepackage{circuitikz}
\usepackage{pgfplots}
\usetikzlibrary{shapes,arrows}
\usetikzlibrary{arrows}
\usetikzlibrary{patterns,fadings}
 \usetikzlibrary{decorations.pathreplacing}
 \usetikzlibrary{snakes}
 \usetikzlibrary{spy}
 \usepackage{setspace}

 \usetikzlibrary{decorations.pathreplacing}
\setkeys{Gin}{width=\linewidth,totalheight=\textheight,keepaspectratio}
\graphicspath{{graphics/}}

%%
% Additional
\usepackage{units}
\usepackage{amsmath,amsfonts,amsthm} % Math packages
\usepackage{mathtools}% http://ctan.org/pkg/mathtools
%\usepackage{mparhack}
\usepackage{sectsty} % Allows customizing section commands
%\usepackage[dvipsnames]{xcolor}
\usepackage{pgf,tikz}
%\usepackage{pgfplots}
%\usetikzlibrary{shapes,arrows}
%\usetikzlibrary{patterns,fadings}
%\usetikzlibrary{arrows}
% \usetikzlibrary{decorations.pathreplacing}
% \usetikzlibrary{snakes}
 %\usetikzlibrary{spy}
 %\usepackage{setspace}
% \usepackage{3dplot}
% \usepackage{cancel}
%\usepackage{physymb}
%\usepackage{braket}
%\usepackage{verbatim}
%\usepackage[x11names]{xcolor}                     %Additional colors
%\usepackage{euler}  



% The fancyvrb package lets us customize the formatting of verbatim
% environments.  We use a slightly smaller font.
\usepackage{fancyvrb}
\fvset{fontsize=\normalsize}


\begin{document}
\pagenumbering{gobble}

\doublespacing
\textbf{IB Physics }                        %%%(class number and section) 
 \hfill                             %%%(date of test)
${\bf Name: } \underline{\hspace{2.5in}}$

\begin{centering}
\vspace{1cm}
\textbf{Exam 2}\\
\end{centering}

\vspace{1cm}
$$E=-\frac{\Delta V}{\Delta x}$$
$$C=\frac{Q}{V}$$
$$I=\frac{\Delta Q}{\Delta t}$$
   $$V=IR$$
   $$P=IV$$
   $$\text{Parallel:}\hspace{1cm} \frac{1}{R_{eq}}=\frac{1}{R_1}+\frac{1}{R_2}$$
   $$\text{Series:}\hspace{1cm} {R_{eq}}={R_1}+{R_2}$$
   $$\overrightarrow{F}_B=q\overrightarrow{v}\times\overrightarrow{B}$$
    $$\overrightarrow{F}_B=\overrightarrow{IL}\times\overrightarrow{B}$$
    $$B_{wire}=\frac{\mu_0 I}{2\pi r}$$
\begin{table}[htbp]
\begin{center}
\footnotesize
\begin{tabular}{lllll}
\toprule
 Description              & Symbol          & Quantity                                                                \\
\midrule
  Gravitational Constant   & $G$             & $6.67 \times 10^{-11} \nicefrac{ \text{N}\cdot\text{m}^2}{\text{kg}^2}$ \\
    Electrostatic Constant   & $k_e$           & $8.99 \times 10^{9} \nicefrac{ \text{N}\cdot\text{m}^2}{\text{C}^2}$    \\
    Boltzmann's Constant     & $k_B$           & $1.38 \times 10^{-23} \nicefrac{ \text{J}}{\text{K}}$                   \\
    Avogado's Number         & $N_A$           & $6.02 \times 10^{23} $                                                  \\
    Plank's Constant         & $h$             & $6.63 \times 10^{-34}  \text{J}\cdot\text{s}$                           \\
    Speed of Light           & $c$             & $3.0 \times 10^{8} \nicefrac{ \text{m}}{\text{s}}$                      \\
    Fundamental Charge       & $e$             & $1.6 \times 10^{-19} \text{C}$                                          \\
    Mass of the Electron     & $m_e$           & $9.1 \times 10^{-31} \text{kg}$                                         \\
    Mass of Proton           & $m_p$           & $1.7 \times 10^{-27} \text{kg}$                                         \\ 
    Gas Constant             & $R$             & $8.31 \nicefrac{ \text{ J}}{\text{mole}\cdot\text{K}}$                  \\
    Vacuum Permativity       & $\varepsilon_0$ & $8.85 \times 10^{-12} \nicefrac{ \text{F}}{\text{m}}$                    \\
    Vacuum Permeablity       & $\mu_0$         & $4\pi \times 10^{-7} \nicefrac{ \text{T}\cdot\text{m}}{\text{A}}$       \\
    Bohr Radius              & $a_0$           & $0.53 \times 10^{-10} \text{m}$                                         \\
    Fine Structure  Constant & $\alpha$        & $\nicefrac{1}{137}$                                              \\ 

\bottomrule
\end{tabular}
\end{center}
  \caption{A list of physical quantities with SI units and dimensions.}
  \label{tab:font-sizes}
\end{table}


\newpage

The first question of the exam is worth 30 points.

 $\bf{1)}$ Consider the following circuit
 

 \begin{circuitikz} 
 \draw (2,0) to[battery, l=$200V$] (0,0);
 \draw (2,0) -- (2,6);
  \draw (0,4) -- (0,6);
 \draw (0,2)     to[resistor, l=$100\Omega$] (0,4);
      \draw (2,4) to[resistor, l=$200\Omega$] (0,4);
      \draw (2,6) to[resistor, l=$400\Omega$] (0,6);
      \draw (0,0) to[resistor, l=$50\Omega$] (0,2);
\end{circuitikz}


 a.  Determine the equivalent resistance of the circuit.
 

   \vspace{2.5cm}

 
 b.  Determine the current through the $50\Omega$ resistor.
 
  \vspace{2.5cm}
  
 c.  Determine the current through the $200\Omega$ resistor.
 
  \vspace{2.5cm}
 
 d. Determine the voltage drop across the $100\Omega$ resistor.
 
  \vspace{2.5cm}
 
 e. Determine the power dissipated by the $400\Omega$ resistor.
 
  
 
 


  \newpage
  The second question is worth 30 points.  
  
 $\bf{2)}$  Consider a magnetic field interacting with a loop of current.  The loop is a 4x6 cm rectangle.  \\
 The wire contains $10^{18}$ free moving electrons.  The magnetic field is $B=0.050$ Tesla.  The current is  $I=1.6$ Amperes.
 
  \tikzset{->-/.style={decoration={
  markings,
  mark=at position #1 with {\arrow{latex}}},postaction={decorate}}}

 
 $$\begin{tikzpicture}[scale=1]
\draw[very thick,->-=.5] (0,0) -- (2,0);
\draw[very thick,->-=.5] (2,0) -- (2,3)node[midway, anchor=west]{$I$};
\draw[very thick,->-=.5] (2,3) -- (0,3) ;
\draw[very thick,->-=.5] (0,3) -- (0,0);
\draw[<-] (-1,0.3)--(-0.2,0.3);
\draw[<-] (-1,0.5)--(-0.2,0.5);
\draw[<-] (-1,0.7)--(-0.2,0.7);
\draw[<-] (-1,0.9)--(-0.2,0.9);
\draw[<-] (-1,1.1)--(-0.2,1.1);
\draw[<-] (-1,1.3)--(-0.2,1.3);
\draw[<-] (-1,1.5)--(-0.2,1.5);
\draw[<-] (-1,1.7)--(-0.2,1.7);
\draw[<-] (-1,1.9)--(-0.2,1.9);
\draw[<-] (-1,2.1)--(-0.2,2.1);
\draw[<-] (-1,2.3)--(-0.2,2.3);
\draw[<-] (-1,2.5)--(-0.2,2.5);
\draw[<-] (-1,2.7)--(-0.2,2.7);
\draw[<-] (-1,2.9)--(-0.2,2.9);

\draw (-1,0.6) node[anchor=east]{$B$};

\begin{scope}[scale={1}, shift={(-0.4,2.4)}]
%\draw (0,0) circle(1.8mm);
%\fill (0,0) circle(0.3mm);
%\draw (-0.2,0) node[anchor=east]{$F_1$};
\end{scope}

\begin{scope}[scale={1}, shift={(2.4,2.4)}]
%\draw (0,0) circle(1.8mm);
%\fill (0,0) circle(0.3mm);
%\draw (0.2,0) node[anchor=west]{$F_2$};
\begin{scope}[scale={1}, rotate={(45)}]
%\draw(-0.18,0) -- (0.18,0);
%\draw(0,-0.18) -- (0,0.18);
\end{scope}
\end{scope}
\end{tikzpicture}
$$ 
  
 a.  Determine the direction of magnetic force on each section of the loop.
   \vspace{3.5cm}
 
 b.  Determine the magnitude of force on each section of the loop.
   \vspace{4.5cm}
 
 c.  Describe the structure of the magnetic field created by the loop.
   \vspace{4.5cm}
 
 d.  Determine the subsequent motion of the loop if it is free to move.
 

  \newpage

Question three is worth 30 points.

  $\bf{3)}$  Consider a charged capacitor that holds $60\times10^{-3}$ Coulombs with $12$ Volts of potential.   The capacitor is connected in series with a $300\  \Omega$ resistor.  The capacitor begins discharging at $t=0$.  \vspace{0.5cm}
  
  a.  Draw the circuit described above.
    \vspace{5.5cm}
  
  b.  Draw a graph of the current as a function of time, $I(t)$.  Include the value of the initial current.
    \vspace{5.5cm}
  
  c.  Explain how the capacitor functions as a battery in this system.
    \vspace{5.5cm}
 
  
  \newpage

Question four is worth 10 points.

  $\bf{4)}$  Two long straight wires, separated by 50 cm, run parallel and carry current in opposite directions.  \vspace{0.5cm}
  
Describe the magnetic force between the wires.
\vspace{5cm}

Explain how these wires could be used to define the Ampere.
  
 
\end{document}